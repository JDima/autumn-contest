\documentclass[a4]{article}
\pagestyle{myheadings}
\setlength{\parindent}{5ex}
%%%%%%%%%%%%%%%%%%%
% Packages/Macros %
%%%%%%%%%%%%%%%%%%%
\usepackage{mathrsfs}


\usepackage{fancyhdr}
\pagestyle{fancy}
\lhead{}
\chead{}
\rhead{}
\lfoot{}
\cfoot{} 
\rfoot{\normalsize\thepage}
\renewcommand{\headrulewidth}{0pt}
\renewcommand{\footrulewidth}{0pt}
\newcommand{\RomanNumeralCaps}[1]
    {\MakeUppercase{\romannumeral #1}}

\usepackage{amssymb,latexsym}  % Standard packages
\usepackage[utf8]{inputenc}
\usepackage[russian]{babel}
\usepackage{MnSymbol}
\usepackage{mathrsfs}
\usepackage{amsmath,amsthm}
\usepackage{indentfirst}
\usepackage{graphicx}%,vmargin}
\usepackage{graphicx}
\graphicspath{{pictures/}} 
\usepackage{verbatim}
\usepackage{color}
\usepackage{color,colortbl}
\usepackage[nottoc,numbib]{tocbibind}
\usepackage{float}
\usepackage{multirow}
\usepackage{hhline}

\usepackage{listings}
\definecolor{codegreen}{rgb}{0,0.6,0}
\definecolor{codegray}{rgb}{1,1,1}
\definecolor{codepurple}{rgb}{0.58,0,0.82}
\definecolor{backcolour}{rgb}{0.95,0.95,0.92}
 
\lstdefinestyle{mystyle}{
    backgroundcolor=\color{backcolour},   
    commentstyle=\color{codegreen},
    keywordstyle=\color{magenta},
    numberstyle=\tiny\color{codegray},
    stringstyle=\color{codepurple},
    basicstyle=\footnotesize,
    breakatwhitespace=false,         
    breaklines=true,                 
    captionpos=b,                    
    keepspaces=true,                 
    numbers=left,                    
    numbersep=5pt,                  
    showspaces=false,                
    showstringspaces=false,
    showtabs=false,                  
    tabsize=2
}
 
\lstset{style=mystyle}

\usepackage{url}
\urldef\myurl\url{foo%.com}
\def\UrlBreaks{\do\/\do-}
\usepackage{breakurl}
\Urlmuskip=0mu plus 1mu



\DeclareGraphicsExtensions{.pdf,.png,.jpg}% -- настройка картинок

\usepackage{epigraph} %%% to make inspirational quotes.
\usepackage[all]{xy} %for XyPic'a
\usepackage{color} 
\usepackage{amscd} %для коммутативных диграмм
%\usepackage[colorlinks,urlcolor=red]{hyperref}

%\renewcommand{\baselinestretch}{1.5}
%\sloppy
%\usepackage{listings}
%\lstset{numbers=left}
%\setmarginsrb{2cm}{1.5cm}{1cm}{1.5cm}{0pt}{0mm}{0pt}{13mm}


\newtheorem{Lemma}{Лемма}[section]
\newtheorem{Proposition}{Предложение}[section]
\newtheorem{Theorem}{Теорема}[section]
\newtheorem{Corollary}{Следствие}[section]
\newtheorem{Remark}{Замечание}[section]
\newtheorem{Definition}{Определение}[section]
\newtheorem{Designations}{Обозначение}[section]




%%%%%%%%%%%%%%%%%%%%%%% 
%Подготовка оглавления% 
%%%%%%%%%%%%%%%%%%%%%%% 
\usepackage[titles]{tocloft}
\renewcommand{\cftdotsep}{2} %частота точек
\renewcommand\cftsecleader{\cftdotfill{\cftdotsep}}
\renewcommand{\cfttoctitlefont}{\hspace{0.38\textwidth} \LARGE\bfseries} 
\renewcommand{\cftsecaftersnum}{.}
\renewcommand{\cftsubsecaftersnum}{.}
\renewcommand{\cftbeforetoctitleskip}{-1em} 
\renewcommand{\cftaftertoctitle}{\mbox{}\hfill \\ \mbox{}\hfill{\footnotesize Стр.}\vspace{-0.5em}} 
%\renewcommand{\cftchapfont}{\normalsize\bfseries \MakeUppercase{\chaptername} } 
%\renewcommand{\cftsecfont}{\hspace{1pt}} 
\renewcommand{\cftsubsecfont}{\hspace{1pt}} 
%\renewcommand{\cftbeforechapskip}{1em} 
\renewcommand{\cftparskip}{3mm} %определяет величину отступа в оглавлении
\setcounter{tocdepth}{5} 
\renewcommand{\listoffigures}{\begingroup %добавляем номер в список иллюстраций
\tocsection
\tocfile{\listfigurename}{lof}
\endgroup}
\renewcommand{\listoftables}{\begingroup %добавляем номер в список иллюстраций
\tocsection
\tocfile{\listtablename}{lot}
\endgroup}


%\renewcommand{\thelikesection}{(\roman{likesection})}
%%%%%%%%%%%
% Margins %
%%%%%%%%%%%
\addtolength{\textwidth}{0.7in}
\textheight=630pt
\addtolength{\evensidemargin}{-0.4in}
\addtolength{\oddsidemargin}{-0.4in}
\addtolength{\topmargin}{-0.4in}

%%%%%%%%%%%%%%%%%%%%%%%%%%%%%%%%%%%
%%%%%%Переопределение chapter%%%%%% 
%%%%%%%%%%%%%%%%%%%%%%%%%%%%%%%%%%%
\newcommand{\empline}{\mbox{}\newline} 
\newcommand{\likechapterheading}[1]{ 
\begin{center} 
\textbf{\MakeUppercase{#1}} 
\end{center} 
\empline} 

%%%%%%%Запиливание переопределённого chapter в оглавление%%%%%% 
\makeatletter 
\renewcommand{\@dotsep}{2} 
\newcommand{\l@likechapter}[2]{{\bfseries\@dottedtocline{0}{0pt}{0pt}{#1}{#2}}} 
\makeatother 
\newcommand{\likechapter}[1]{ 
\likechapterheading{#1} 
\addcontentsline{toc}{likechapter}{\MakeUppercase{#1}}} 




\usepackage{xcolor}
\usepackage{hyperref}
\definecolor{linkcolor}{HTML}{000000} % цвет ссылок
\definecolor{urlcolor}{HTML}{AA1622} % цвет гиперссылок
 
\hypersetup{pdfstartview=FitH,  linkcolor=linkcolor,urlcolor=urlcolor, colorlinks=true}

%%%%%%%%%%%%
% Document %
%%%%%%%%%%%%

%%%%%%%%%%%%%%%%%%%%%%%%%%%%%
%%%%%%главы -- section*%%%%%%
%%%%section -- subsection%%%%
%subsection -- subsubsection%
%%%%%%%%%%%%%%%%%%%%%%%%%%%%%
\def \newstr {\medskip \par \noindent} 
\begin{document}



\section*{Задача 1}
\label{sec:orgb62fe60}
\subsection*{Постановка}
\label{sec:org37954e9}
Студенты готовят посвят для первокурсников. У них имеется $k$ видов алкогольной продукции, каждый вид характеризуется концентрацией спирта $a_i/1000$. На праздник студенты решили приготовить коктель "Бомба" концентрацией алкоголя ровно $n/1000$. Они хотят, чтобы в "Бомбе" было целое число литров алкоголя каждого вида (алкогольной продукции некоторых видов может и вовсе не быть в коктеле). Кроме того, они хотят минимизировать общий объем алкоголя в "Бомбе". 

Концентрацией спирта в алкоголе называется отношение объема спирта к общему объему напитка. При смешивании объем спирта в напитке равен суммарному объему спирта в смешиваемых компонентах; объем напитка также равен сумме объемов напитка в смешиваемых компонентах.

Найдите минимальное натуральное количество литров алкоголя, необходимое для получения "Бомбы" с концентрацией спирта ровно $n/1000$.
\subsection*{Входные данные}
\label{sec:orgc51833b}
В первой строке входных данных содержится число $n$, $0\leq n \leq 1000$.

В следующей строке содержится $k$ чисел $a_1..a_k$, $0\leq a_i \leq 1000$
\subsection*{Выходные данные}
\label{sec:org91cd1c2}
Выведите минимальное натуральное количество литров алкоголя, необходимое для получения "Бомбы" с концентрацией спирта ровно $n/1000$, или -1, если это сделать невозможно.

\subsection*{Пример 1}
\label{sec:org1b720b0}

\begin{table}[H]
\begin{center}
\begin{tabular}{|m{4cm}|m{4cm}|}
\hline
Входные данные & Выходные данные \\ \hline
400

300 100 550 600
&
3
\\ \hline
\end{tabular}
\end{center}
\end{table}

\subsection*{Пример 2}
\label{sec:org2aeecb4}

\begin{table}[H]
\begin{center}
\begin{tabular}{|m{4cm}|m{4cm}|}
\hline
Входные данные & Выходные данные \\ \hline
50

125  25
&
4 
\\ \hline
\end{tabular}
\end{center}
\end{table}

\pagebreak
\section*{Задача 2}
\label{sec:orgef181bd}
\subsection*{Постановка}
\label{sec:orgad8a20e}
Диме дана последовательность чисел $a_1..a_n$. Учитель сказал зачеркнуть число $a_i$ и найти подпоследовательность с максимальной суммой, не содержащую ни одного зачёркнутого числа. Сумму чисел в пустой подпоследовательности считать равной 0. После повторять действия пока не будут зачеркнуты все числа. Помогите Диме выполнить задание учителя.

\subsection*{Входные данные}
\label{sec:orgc51833b}
В первой строке записаны $n$ чисел $a_1..a_n$.

$0\leq n \leq 100000$, $0\leq a_i \leq 10^9$

Во второй строчке записано последовательность в которой зачеркиваются числа.
\subsection*{Выходные данные}
\label{sec:orgf9da829}
Выведите $n$ чисел каждое из которых - максимальная сумма на подпоследовательности, после выполнения очередного действия. 

\subsection*{Пример 1}
\label{sec:orgd7d348d}

\begin{table}[H]
\begin{center}
\begin{tabular}{|m{4cm}|m{4cm}|}
\hline
Входные данные & Выходные данные \\ \hline
1 3 2 5

3 4 1 2
&
5
4
3
0
\\ \hline
\end{tabular}
\end{center}
\end{table}
\subsection*{Пример 2}
\label{sec:orgd7d348d}

\begin{table}[H]
\begin{center}
\begin{tabular}{|m{4cm}|m{4cm}|}
\hline
Входные данные & Выходные данные \\ \hline
1 2 3 4 5

4 2 3 5 1
&
6
5
5
1
0
\\ \hline
\end{tabular}
\end{center}
\end{table}
\pagebreak
\section*{Задача 3}
\label{sec:org570b899}
\subsection*{Постановка}
\label{sec:orga2b5149}
На доске нарисованы точки. Стоимость соединения двух точек равняется Манхэттенскому расстоянию между ними.
Найдите минимальную стоимость соединения всех точек.  
\subsection*{Входные данные}
\label{sec:orgeb4908d}
$n$ строк содержащих два числа $x$ и $y$ - координаты точки.

$0\leq n \leq 1000$,$-10^6 \leq x \leq 10^6$,$-10^6 \leq y \leq 10^6$.
\subsection*{Выходные данные}
\label{sec:orged795e8}
Выведите минимальную стоимость соединения всех точек
\subsection*{Пример 1}
\label{sec:org6a26c04}

\begin{table}[H]
\begin{center}
\begin{tabular}{|m{4cm}|m{4cm}|}
\hline
Входные данные & Выходные данные \\ \hline
0 0

2 2

3 10

5 2

7 0
&
20 
\\ \hline
\end{tabular}
\end{center}
\end{table}

\subsection*{Пример 2}
\label{sec:orge96f7c4}

\begin{table}[H]
\begin{center}
\begin{tabular}{|m{4cm}|m{4cm}|}
\hline
Входные данные & Выходные данные \\ \hline
3 12 

-2 5 

-4 1
&
18
\\ \hline
\end{tabular}
\end{center}
\end{table}
\subsection*{Пример 3}
\label{sec:orge96f7c4}

\begin{table}[H]
\begin{center}
\begin{tabular}{|m{4cm}|m{4cm}|}
\hline
Входные данные & Выходные данные \\ \hline
0 0
&
0
\\ \hline
\end{tabular}
\end{center}
\end{table}
\pagebreak
\section*{Задача 4}
\label{sec:org570b899}
\subsection*{Постановка}
\label{sec:orga2b5149}
Дана последовательность точек на плоскости. Значения отсортированы по координате $x$. Так же дано целое число $k$.
Найдите максимальное значение выражения $y_i + y_j + |x_i-x_j|$, при условии что  $|x_i-x_j|\leq k$. Гарантируется, что существует хотя бы одна пара точек, удовлетворяющая ограничению. 
\subsection*{Входные данные}
\label{sec:orgeb4908d}
$n$ строк содержащих два числа $x$ и $y$ - координаты точки.

$2\leq n \leq 10^5$,$-10^8 \leq x \leq 10^8$,$-10^8 \leq y \leq 10^8$, $0\leq k \leq 10^8$
\subsection*{Выходные данные}
\label{sec:orged795e8}
Выведите максимальное значение выражения.
\subsection*{Пример 1}
\label{sec:org6a26c04}

\begin{table}[H]
\begin{center}
\begin{tabular}{|m{4cm}|m{4cm}|}
\hline
Входные данные & Выходные данные \\ \hline
1 3

2 0

5 10

6 -10
&
1 
\\ \hline
\end{tabular}
\end{center}
\end{table}

\subsection*{Пример 2}
\label{sec:orge96f7c4}

\begin{table}[H]
\begin{center}
\begin{tabular}{|m{4cm}|m{4cm}|}
\hline
Входные данные & Выходные данные \\ \hline
0 0 

0 3 

9 2
&
3
\\ \hline
\end{tabular}
\end{center}
\end{table}

\pagebreak

\section*{Задача 5}
\label{sec:orgb62fe60}
\subsection*{Постановка}
\label{sec:org37954e9}
В стране Турляндии $n$ городов и $m$ двунаправленных дорог. Все междугородние дороги платные. Также в Турляндии есть интересная особенность, нельзя останавливаться в первом по пути городе. Это означает, что нужно проехать из города $x$ в город $y$ и не останавливаясь в нем поехать в город $z$. Цена такой поездки будет равняться $(w_{xy}+w_{yz})^2$, где $w_{xy}$ - цена проезда между городами $x$ и $y$, $w_{yz}$ - цена проезда между городами $y$ и $z$. Для каждого города найдите, можно ли добраться до него из города 1, и какое наименьшее количество денег необходимо потратить на такой путь.
\subsection*{Входные данные}
\label{sec:orgc51833b}
В первой строке находятся два целых числа $n$ - количество городов, $m$ - количество дорог.

В следующих $m$ строках находятся по три целых числа $x$ и $y$ - номера городов , $w_{xy}$ - цена проезда между городами. Гарантируется что если есть строка $x$ $y$ $w_{xy}$ то не будет строки $y$ $x$ $w_{yx}$.

$2\leq n \leq 10^5$

$1 \leq m \leq min((n^2-n)/2;2*10^5)$

$-10^8 \leq y \leq 10^8$

$0\leq k \leq 10^8$

 $1 \leq w_{xy} \leq 50$
\subsection*{Выходные данные}
\label{sec:org91cd1c2}
Для каждого города выведите одно целое число. Если нет корректного пути из 1 города выведите -1. Иначе выведите минимальное необходимое количество денег, чтобы добраться из 1 города.

\subsection*{Пример 1}
\label{sec:org1b720b0}

\begin{table}[H]
\begin{center}
\begin{tabular}{|m{4cm}|m{4cm}|}
\hline
Входные данные & Выходные данные \\ \hline
5 6

1 2 3

2 3 4

3 4 5

4 5 6

1 5 1

2 4 2
&
0 98 49 25 114 
\\ \hline
\end{tabular}
\end{center}
\end{table}

\subsection*{Пример 2}
\label{sec:org2aeecb4}

\begin{table}[H]
\begin{center}
\begin{tabular}{|m{4cm}|m{4cm}|}
\hline
Входные данные & Выходные данные \\ \hline
3 2

1 2 1

2 3 2
&
0 -1 9 
\\ \hline
\end{tabular}
\end{center}
\end{table}

\pagebreak

\end{document}